\documentclass[a4paper, 9pt]{extarticle}

\usepackage{amsmath, amsfonts, amsthm, amssymb}
\usepackage[left=.5cm,top=1.5cm,right=.5cm,bottom=2cm]{geometry}
\usepackage{multicol}

\newenvironment{enumx}{\begin{enumerate} \setlength{\itemsep}{0pt} \setlength{\parskip}{0pt} \setlength{\parsep}{0pt}}{\end{enumerate}}
\newenvironment{itemx}{\begin{itemize} \setlength{\itemsep}{0pt} \setlength{\parskip}{0pt} \setlength{\parsep}{0pt}}{\end{itemize}}

\usepackage[usenames,dvipsnames]{color}
\definecolor{wordred}{rgb}{0.9,  0,    0}
\definecolor{wordblu}{rgb}{0,    0.45, 0.75}
\newcommand{\farbea}[1]{{\color{wordred} {#1}}}
\newcommand{\farbeb}[1]{{\color{wordblu} {#1}}}
\newcommand{\spk}[3]{\stepcounter{definicja} \textbf{\farbeb{#1}}: {#2} \farbea{(#3)}}
\newcommand{\kapt}[1]{\noindent \stepcounter{kapcz} (\arabic{kapcz}) \dotfill \textbf{[#1]}}

\newcounter{definicja}
\newcounter{kapcz}

%\usepackage[parfill]{parskip}
%usepackage[none]{hyphenat}

% \setlength{\parskip}{0.25\baselineskip}
%\relpenalty=10000
%\binoppenalty=10000

\usepackage{Alegreya}
%\usepackage{microtype}

% Język
\usepackage[polish]{babel}
\usepackage[utf8]{inputenc}
\usepackage[T1]{fontenc}
\selectlanguage{polish}

\author{L. Löwenherz}
\title{Matematyka}

\begin{document} 
\maketitle

\begin{multicols}{2}
\stepcounter{definicja} 
\kapt{Algebra abstrakcyjna}\begin{enumx}\item [\arabic{definicja}.] \spk{assoziativ}{łączny}{associative} %katalgebra
\item [\arabic{definicja}.] \spk{}{automorfizm}{automorphism} %katalgebra
\item [\arabic{definicja}.] \spk{Endomorphismus, m}{endomorfizm}{} %katalgebra
\item [\arabic{definicja}.] \spk{}{epimorfizm}{} %katalgebra
\item [\arabic{definicja}.] \spk{Gruppe, f}{grupa}{group} %katalgebra
\item [\arabic{definicja}.] \spk{}{homomorfizm}{} %katalgebra
\item [\arabic{definicja}.] \spk{}{izomorfizm}{} %katalgebra
\item [\arabic{definicja}.] \spk{kommutativ}{przemienny}{commutative} %katalgebra
\item [\arabic{definicja}.] \spk{Körper, m}{ciało}{} %katalgebra
\item [\arabic{definicja}.] \spk{}{moduł}{} %katalgebra
\item [\arabic{definicja}.] \spk{}{monomorfizm}{} %katalgebra
\item [\arabic{definicja}.] \spk{}{normalizator}{} %katalgebra
\item [\arabic{definicja}.] \spk{Operation, f}{działanie}{} %katalgebra
\item [\arabic{definicja}.] \spk{}{orbita}{} %katalgebra
\item [\arabic{definicja}.] \spk{}{pierścień}{} %katalgebra
\item [\arabic{definicja}.] \spk{}{produkt prosty}{direct product} %katalgebra
\item [\arabic{definicja}.] \spk{}{stabilizator}{} %katalgebra
\item [\arabic{definicja}.] \spk{}{warstwa}{coset} %katalgebra
\item [\arabic{definicja}.] \spk{Zentrum, n}{centrum}{} %katalgebra
\end{enumx}
\kapt{Algebra liniowa}\begin{enumx}\item [\arabic{definicja}.] \spk{}{}{dot/inner product} %katliniowa
\item [\arabic{definicja}.] \spk{Spur, f}{ślad}{trace} %katliniowa
\end{enumx}
\kapt{Analiza}\begin{enumx}\item [\arabic{definicja}.] \spk{differenzierbar}{różniczkowalny}{differentiable} %katanaliza
\item [\arabic{definicja}.] \spk{divergent}{rozbieżny}{divergent} %katanaliza
\item [\arabic{definicja}.] \spk{}{dyfeomorfizm}{} %katanaliza
\item [\arabic{definicja}.] \spk{}{}{extremum} %katanaliza
\item [\arabic{definicja}.] \spk{gleichmäßig}{jednostajnie}{uniformly} %katanaliza
\item [\arabic{definicja}.] \spk{Gleichung, f}{}{} %katanaliza
\item [\arabic{definicja}.] \spk{}{granica}{limit} %katanaliza
\item [\arabic{definicja}.] \spk{Integral, n}{całka}{integral} %katanaliza
\item [\arabic{definicja}.] \spk{konvergent}{zbieżny}{convergent} %katanaliza
\item [\arabic{definicja}.] \spk{Logarithmus, m}{logarytm}{} %katanaliza
\item [\arabic{definicja}.] \spk{periodisch}{}{periodical} %katanaliza
\item [\arabic{definicja}.] \spk{Potenzreihe, f}{szereg potęgowy}{power series} %katanaliza
\item [\arabic{definicja}.] \spk{Reihe, f}{szereg}{series} %katanaliza
\item [\arabic{definicja}.] \spk{}{}{vector field} %katanaliza
\end{enumx}
\kapt{Geometria}\begin{enumx}\item [\arabic{definicja}.] \spk{Ball, m}{kula}{ball} %katgeometria
\item [\arabic{definicja}.] \spk{Dreieck, n}{trójkąt}{triangle} %katgeometria
\item [\arabic{definicja}.] \spk{Durchmesser, ?}{średnica}{diameter} %katgeometria
\item [\arabic{definicja}.] \spk{Ebene, f}{płaszczyzna}{plane} %katgeometria
\item [\arabic{definicja}.] \spk{Gerade, f}{prosta}{line} %katgeometria
\item [\arabic{definicja}.] \spk{Quadrat, n}{kwadrat}{square} %katgeometria
\item [\arabic{definicja}.] \spk{Rechteck, n}{prostokąt}{rectangle} %katgeometria
\item [\arabic{definicja}.] \spk{Sehne, f}{cięciwa}{chord} %katgeometria
\item [\arabic{definicja}.] \spk{Torus, m}{torus}{torus} %katgeometria
\item [\arabic{definicja}.] \spk{Würfel, m}{sześcian}{cube} %katgeometria
\end{enumx}
\kapt{Rachunek prawdopodobieństwa}\begin{enumx}\item [\arabic{definicja}.] \spk{Erwartungswert, m}{wartość oczekiwana}{expected value} %katpstwo
\item [\arabic{definicja}.] \spk{Kovarianz, f}{kowariancja}{covariance} %katpstwo
\item [\arabic{definicja}.] \spk{Varianz, f}{wariancja}{variance} %katpstwo
\item [\arabic{definicja}.] \spk{Verteilungsfunktion, f}{dystrybuanta}{cumulative distribution f.} %katpstwo
\item [\arabic{definicja}.] \spk{Verteilung, f}{rozkład p-stwa}{distribution} %katpstwo
\item [\arabic{definicja}.] \spk{Wahrscheinlichkeit, f}{prawdopodobieństwo}{probability} %katpstwo
\item [\arabic{definicja}.] \spk{Zufallsvariable, f}{zmienna losowa}{random variable} %katpstwo
\end{enumx}
\kapt{Teoria mnogości}\begin{enumx}\item [\arabic{definicja}.] \spk{Funktion, f}{funkcja}{function} %katzbiory
\item [\arabic{definicja}.] \spk{Menge, f}{zbiór}{set} %katzbiory
\item [\arabic{definicja}.] \spk{Relation, f}{relacja}{relation} %katzbiory
\item [\arabic{definicja}.] \spk{}{surjekcja}{} %katzbiory
\end{enumx}
\kapt{Topologia}\begin{enumx}\item [\arabic{definicja}.] \spk{abgeschlossen}{domknięty}{closed} %kattopologia
\item [\arabic{definicja}.] \spk{Homöomorphismus, m}{homeomorfizm}{homeomorphism} %kattopologia
\item [\arabic{definicja}.] \spk{kompakt}{zwarty}{compact} %kattopologia
\item [\arabic{definicja}.] \spk{offen}{otwarty}{open} %kattopologia
\item [\arabic{definicja}.] \spk{separabel}{ośrodkowy}{separable} %kattopologia
\item [\arabic{definicja}.] \spk{stetig}{ciągły}{continuous} %kattopologia
\item [\arabic{definicja}.] \spk{Topologie, f}{topologia}{topology} %kattopologia
\item [\arabic{definicja}.] \spk{Trennungsaxiom, n}{aksjomat oddzielania}{separation axiom} %kattopologia
\item [\arabic{definicja}.] \spk{vollständig}{zupełny}{complete} %kattopologia
\item [\arabic{definicja}.] \spk{zusammenhängen}{spójny}{connected} %kattopologia
\end{enumx}
\vfill
\newpage

\kapt{Śmietnik dla leniwych}
\begin{enumx}
\item [\arabic{definicja}.] \spk{Fixpunkt, m}{punkt stały}{fixed point} %katnokat
\item [\arabic{definicja}.] \spk{Permutation, f}{permutacja}{permutation} %katnokat

\end{enumx}

\end{multicols}
\end{document}